\documentclass[11pt]{article}


\usepackage{amsmath}
\usepackage{amssymb}
\usepackage{graphicx,tikz,pgfplots}
\usepackage{epstopdf}
\usepackage{epsfig}
\newtheorem{theorem}{Theorem}[section]
\newtheorem{corollary}{Corollary}[theorem]
\newtheorem{lemma}[theorem]{Lemma}
\newtheorem{definition}{Definition}
\setlength{\oddsidemargin}{0in}  % set margins
\setlength{\evensidemargin}{0in} %
\setlength{\textwidth}{6.5in}    %
\setlength{\textheight}{9.0in}   %
\setlength{\topmargin}{-0.5in}   %
\parskip=5pt
\newcommand{\norm}[1]{\left\lVert#1\right\rVert}
\usepgfplotslibrary{patchplots}
\usetikzlibrary{patterns, positioning, arrows}

\newcommand\mat{{\sf MATLAB}}
\newcommand{\mb}[1]{\hbox{\textbf{#1}}}
\newcommand{\rmmax}{\mathrm{max}}
\newcommand{\half}{\frac{1}{2}}

\renewcommand{\labelenumii}{\alph{enumii}.}
\renewcommand{\labelenumi}{\textbf{\arabic{enumi}}}

\begin{document}


\begin{center}
{\large \bf	Dynamical systems reading group \\[1ex]
			Vector fields and smooth flows
\\[2ex]
	   	{\it Date: Sept. 1st, 2017} \hfill \\[-1.5ex]	  
 	\hrulefill \\
				
}
\end{center}


%%================================================================%%
\section{Review of existence and uniqueness theorem }
%%================================================================%%
%%----------------------------------------------------------------%%

Recall the existence and uniqueness theorem for initial value problems. 
Let an initial value problem be given by 
\begin{align}
		\frac{du}{dt} &= f(u) \\
		u(0) &= u_0 \in \mathbb{R}^m. 
\end{align}
If $f:\mathbb{R}^m \to \mathbb{R}^m$ is in $C^1(U)$ for an open subset $U \subset \mathbb{R}^m$, then a unique solution $u(t)$
exists at least for a short time $-\epsilon < t < \epsilon$.

\section{Vector fields on differentiable manifolds}

The union of the tangent spaces at each $x \in M$ is also called
the tangent bundle $TM := \cup_{x \in M} T_x M$. The \emph{vector field}
is part of the tangent bundle and is given by the set $\left\{\xi(x)\right\}_{x\in M}$
where $\xi(x) \in T_x M$ is the tangent vector at $x$.

\begin{definition}
Let $(U,h)$ be a coordinate chart on the manifold $M$. Then, the vector field    
\end{definition}

The next theorem (Ex. 0.2.1 in the book \cite{katok}) 
describes the role of compactness in 

\bibliographystyle{plain}
\bibliography{bibfile}


\end{document}
