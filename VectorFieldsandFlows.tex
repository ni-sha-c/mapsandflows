\documentclass[11pt]{article}


\usepackage{amsmath}
\usepackage{amssymb,amsthm}
\usepackage{graphicx,tikz,pgfplots}
\usepackage{epstopdf}
\usepackage{epsfig}
\newtheorem{theorem}{Theorem}[section]
\newtheorem{corollary}{Corollary}[theorem]
\newtheorem{lemma}[theorem]{Lemma}
\newtheorem{definition}{Definition}
\setlength{\oddsidemargin}{0in}  % set margins
\setlength{\evensidemargin}{0in} %
\setlength{\textwidth}{6.5in}    %
\setlength{\textheight}{9.0in}   %
\setlength{\topmargin}{-0.5in}   %
\parskip=5pt
\newcommand{\norm}[1]{\left\lVert#1\right\rVert}
\usepgfplotslibrary{patchplots}
\usetikzlibrary{patterns, positioning, arrows}

\newcommand\mat{{\sf MATLAB}}
\newcommand{\mb}[1]{\hbox{\textbf{#1}}}
\newcommand{\rmmax}{\mathrm{max}}
\newcommand{\half}{\frac{1}{2}}

\renewcommand{\labelenumii}{\alph{enumii}.}
\renewcommand{\labelenumi}{\textbf{\arabic{enumi}}}

\begin{document}


\begin{center}
{\large \bf	Dynamical systems reading group \\[1ex]
			S2:Vector fields and smooth flows
\\[2ex]
	   	{\it Date: Sept. 1st, 2017} \hfill \\[-1.5ex]	  
 	\hrulefill \\
				
}
\end{center}


%%================================================================%%
\section{Review of existence and uniqueness theorem }
%%================================================================%%
%%----------------------------------------------------------------%%

Recall the existence and uniqueness theorem for initial value problems. 
\begin{theorem}
Let an initial value problem be given by 
\begin{align}
		\frac{du}{dt} &= f(u) \\
		u(0) &= u_0 \in \mathbb{R}^m. 
\end{align}
If $f:\mathbb{R}^m \to \mathbb{R}^m$ is in $C^1(U)$ for an open subset $U \subset \mathbb{R}^m$, then a unique solution $u(t)$
exists at least for a short time $-\epsilon < t < \epsilon$.
\label{thm:ivp}
\end{theorem}

\section{Vector fields on differentiable manifolds}

The union of the tangent spaces at each $x \in M$ is also called
the tangent bundle $TM := \cup_{x \in M} T_x M$. The \emph{vector field}
is part of the tangent bundle and is given by the set $\left\{\xi(x)\right\}_{x\in M}$
where $\xi(x) \in T_x M$ is the tangent vector at $x$. 

Let $(U,h)$ be a coordinate chart on the $n$-manifold $M$. Then, the coordinates
of an $x \in U \subset M$ are given by $h(x) = s \in \mathbb{R}^n$ or the 
$n$-tuple $(s_1,\cdots,s_n)$. Then, 
starting from an initial $x_0$, the initial value problem for its 
coordinates is:

\begin{align}
\notag \frac{d s_i}{dt} &= v_i(s_1,\cdots,s_n) \\
s(0) &= h(x_0).
\label{eqn:ivp}
\end{align}    

In the above initial value problem, the set of $n$-tuples of real-valued 
functions 
$$\left\{ (v_1(h(x)),v_2(h(x)),\cdots,v_n(h(x))) \right\}_{x \in U}$$ is the vector field on $U$. Due to theorem \ref{thm:ivp}, the above IVP is able
to describe a dynamical system at least for a short time.  
\begin{definition}
A vector field is called \emph{complete} if the solution to 
the IVP in equation \ref{eqn:ivp} exists for all time.  
\end{definition}

In the last note, we made a claim that compactness of a 
manifold makes it easier to describe dynamical systems 
on them. The next theorem (Ex. 0.2.1 in the book \cite{katok}) 
tells us why.

\begin{theorem}
A smooth vector field on a compact manifold is complete.
\end{theorem}

\begin{proof}
\begin{itemize}
\item Let $M$ be a compact $n$-manifold. Then,
$M \subset \left\{(U^{(i)},h^{(i)})\right\}_{i=1}^K$ for a 
finite $K$.

\item For each $x_0^{(i)} \in U^{(i)}, 1 \leq i \leq K$,
the IVP: 
\begin{align}
\notag \frac{d s^i_j}{dt} &= v_j^i(s^i_1,\cdots,s^i_n),\;\; 1 \leq j \leq N \\
s^i(0) &= h^i(x_0).
\label{eqn:ivp}
\end{align}    
has a unique solution $s^i(t) \in h^i(U^i) \subset \mathbb{R}^n$ for 
all time $t \leq t^i$ due to theorem \ref{thm:ivp} since 
$\left\{ v^i_j \right\}_{1\leq j\leq n} \in C^1(\mathbb{R})$ by assumption. 

\item There exists an $s^i(t) \in h^k(U^k), k \neq i$, for some $t < t_i$, 
since each chart is large in some sense. This way, the solution is obtained
for all time, by moving through different coordinate systems.
(this is incomplete). 
\end{itemize}
\end{proof}

\bibliographystyle{plain}
\bibliography{bibfile}


\end{document}
